\documentclass[gradu,emptyfirstpagenumber]{tktltiki}
\usepackage{lmodern}
\usepackage{url}
\usepackage[dvips, bookmarks, colorlinks=false, pdftitle={Tuqqna - Connect Four
game}, pdfauthor={Sami Saada}, pdfsubject={Design Document}, pdfkeywords={PDF,
LaTeX, hyperlinks, hyperref}]{hyperref}

\begin{document}

\title{Tuqqna}
\author{Sami Saada}
\faculty{Faculty of Science}
\department{Department of Computer Science}
\level{Design Document}
\subject{Programming in Python}
\additionalinformation{Exercise group teacher is Lauri Alanko (lealanko)}
\numberofpagesinformation{\numberofpages\ pages}
\keywords{Python, Design document}

\maketitle

\begin{abstract}
Design document of Tuqqna presents the design of Tuqqna, another implementation
of Connect Four game. This document contains some information about the user
interface, how it will be implemented. There is also section of components of
project, which packages, modules and classes there will be.
\end{abstract}

\mytableofcontents

\section{What is Tuqqna}

Tuqqna is another implementation of Connect Four game, which is written in
Python programming language. The name of Tuqqna comes from Kabyle word, which
means connection. It will be fully tested, thanks to Test-Driven Development.
The orginal reason to create Tuqqna is project for Programming in Python course.

\subsection{What Kind of Game is Connect Four}

Connect Four is well-known game, where two players drop buttons and try to
connect four or more button in horizontal, vertical or diagonal lines. The basic
Connect Four game board is 7x6, there are few variants of size. Tuqqna will be
the basic 7x6, but that can be changed too.

Longer and specific description is in Connect Four article in Wikipedia:
\url{http://en.wikipedia.org/wiki/Connect_Four}

\subsection{What's Different}

The basic game mechanic is same as in most of implementations of Connect Four.
Board looks same and rules are same. There is no high scores, but Tuqqna counts
victories and losses of both players in each game session. This provides
opportunity to organize Connect Four tournament with Tuqqna. Tuqqna has own
Artificial Intelligence to play against human players. Of course human versus
human is always an option.

\pagebreak

\section{User Interface}

There will be two user interfaces in Tuqqna, one to use in command line and one
for graphical environment.

\subsection{Command line User Interface}

Command line user interface in Tuqqna will contain two parts. The first part is
for information output, like help file. The second part contains curses user
interface. Curses will be very handy interface for a game like Tuqqna, it
provides all the functionality that is needed.

Curses is part of Python's standard library so it wont add more depencies. Many
highly used command line application, like irssi, uses curses or ncurses, so it
wont be bad choice. Curses contains lots of handy things like input areas and
screen refreshing. Curses can also have mouse support, which isn't bad thing
either.

\subsection{Graphical User Interface}

Graphical user interface in Tuqqna will be written with Tkinter. The reason of
using Tkinter is that is de-facto standard GUI in python. Tuqqna will have
graphical user interface, because it is the most common game interface in these
days.

\subsection{UI Information Description}

Both of user interfaces contains lots of information. There can be selected
players with names. Also an AI can be selected instead of another human player.
UI will always show the current state of board. Player will always see all the
things that it should see.

\pagebreak

\section{Components of Project}

Tuqqna has lots of different kind of components.

\subsection{Packages}

Tuqqna will be build with nested packages. The main package is tuqqna and it has
user interface packages, $core$ and $ai$ packages and unit tests.

User interface packages, $cli$ and $gui$, contain all user interface specific
modules.

Game core package, $core$, contains all core modules of Tuqqna.

Game artificial intelligence package, $ai$, contains all AI relevant modules.

All unit tests are inside of $test$-package. That package has subpackage for
each package of Tuqqna and those packages contains unit test for its relevant
game package.

\subsection{Modules}

There is huge amount of modules in Tuqqna. Each module contains relevant
class(es) for that module. Every main feature of game will have own module:
board, player, rules, etc.

\subsection{Classes}

Tuqqna is created with object-oriented programming, so classes have very
important part to play.

All unit tests will be created in test case classes.

\lastpage

\end{document}

\endinput
